\section{Problem 2 (2 points)}

\subsection*{(a) Which physical frequencies $F_1$ can $f_1$ correspond to if $F_s = 6000Hz$?}
As the sampling frequency need to be twice as high as the frequency after sampling we get from the equation:

\begin{equation*}
    -\frac{F_s}{2}\leq f \leq \frac{F_s}{2}
\end{equation*}

this gives us:

\begin{equation*}
    -3000Hz\leq f \leq 3000Hz
\end{equation*}


\subsection*{(b)  Use Matlab or Python to generate a sequence of length 4 seconds of $x[n]$.}
\importimagewcaption{2b}{Generated sequence of lenght 4 seconds of $x[n]$ where $A=1$, $f_1=$, $F_s = 6000Hz$ and $T=4s$}

\subsection*{(c)  Use the Matlab command soundsc or the Python command sounddevice.play to listen to the harmonic when the normalized frequency $f_1 = 0.3 $and the sampling rate $F_s$ is given by respectively 1000Hz, 3000Hz and 12000Hz.
Comment on what you hear.}

When listening to a higher normalized frequencies the pitch did also sound higher.

\subsection*{(d) Now assume a fixed sampling rate $F_s = 8000Hz$ while the physical frequency $F_1$ is respectively 1000Hz, 3000Hz and 6000Hz. Comment on what you hear. Relate it to the corresponding normalized frequency $f_1$.}

In this case the pitch was diffrent where the highest pitch was 3000 Hz, but then i have difficulty deciding whenever if 1000Hz og 6000Hz are the highest pitch. When i played the tones again with the corresponding normalized frequency $f_1$ the pitch sounded accordingly to the frequencies 6000Hz being the highest and 1000Hz being the lowest.