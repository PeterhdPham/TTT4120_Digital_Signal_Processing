\section{Problem 1 (2 points)}

\subsection*{(a) Sketch \( x[n] \) and \( y[n] \)}
\importimage{1a}

\subsection*{(b) Sketch \( x[n-k] \) for \( k=3 \) and \( k=-3 \).}
\importimage{1b1}
\importimage{1b2}

\subsection*{(c) Sketch \( x[-n] \).}
\importimage{1c}

\subsection*{(d) Sketch \( x[5-n] \).}
\importimage{1d}

\subsection*{(e) Sketch \( x[n] \cdot y[n] \).}
\importimage{1e}

\subsection*{(f) Express the signal \( x[n] \) by using the unit sample sequence \( \delta[n] \).}

\begin{equation*}
    x[n] = 5\delta[n] + 4\delta[n-1] + 3\delta[n-2] + 2\delta[n-3] + 1\delta[n-4]
\end{equation*}


\subsection*{(g) Express the signal \( y[n] \) by using the unit step signal \( u[n] \).}

\begin{equation*}
    y[n] = u[n-2] - u[n-5]
\end{equation*}


\subsection*{(h) Compute the energy of the signal \( x[n] \).}
The energy og the signal $x[n] =  55$ Solved in python

\subsection*{Python Code}

\begin{lstlisting}[language=Python]
import matplotlib.pyplot as plt
import numpy as np

# Problem 1:
def x_n(n):
    if 0 <= n <= 4:
        return 5 - n
    else:
        return 0

def y_n(n):
    if 2 <= n <= 4:
        return 1
    else:
        return 0
    
n_values = np.arange(-5, 10)  # A range of n values
x_values = [x_n(n) for n in n_values]
y_values = [y_n(n) for n in n_values]
plt.figure()


# (a) Sketch x[n] and y[n]
def problem_1a():
    plt.stem(n_values, x_values, label="x[n]")
    plt.stem(n_values, y_values, label="y[n]", markerfmt='ro')
    plt.xlabel('n')
    plt.ylabel('Amplitude')
    plt.legend()
    plt.title('Problem 1a')
    plt.grid(True)
    plt.show()
problem_1a()

# (b) Sketch x[n-k] for k=3 and k=-3
def problem_1b1():
    plt.stem(n_values-3, x_values, label="x[n]")
    plt.stem(n_values, y_values, label="y[n]", markerfmt='ro')
    plt.xlabel('n')
    plt.ylabel('Amplitude')
    plt.legend()
    plt.title('Problem 1b')
    plt.grid(True)
    plt.show() 
problem_1b1()
def problem_1b2():
    plt.stem(n_values+3, x_values, label="x[n]")
    plt.stem(n_values, y_values, label="y[n]", markerfmt='ro')
    plt.xlabel('n')
    plt.ylabel('Amplitude')
    plt.legend()
    plt.title('Problem 1b')
    plt.grid(True)
    plt.show()
problem_1b2()

# (c) Sketch x[-n]
def problem_1c():
    plt.stem(-1*n_values, x_values, label="x[n]")
    plt.stem(n_values, y_values, label="y[n]", markerfmt='ro')
    plt.xlabel('n')
    plt.ylabel('Amplitude')
    plt.legend()
    plt.title('Problem 1c')
    plt.grid(True)
    plt.show()
problem_1c()

# (d) Sketch x[5-n]
def problem_1d():
    plt.stem(-1*n_values+5, x_values, label="x[n]")
    plt.stem(n_values, y_values, label="y[n]", markerfmt='ro')
    plt.xlabel('n')
    plt.ylabel('Amplitude')
    plt.legend()
    plt.title('Problem 1d')
    plt.grid(True)
    plt.show()
problem_1d()

# (e) Sketch x[n] * y[n]
def problem_1e():
    def x_n_y_n(n):
        return x_n(n)*y_n(n)
    xy_values = [x_n_y_n(n) for n in n_values]

    plt.stem(n_values, xy_values, label="x[n]")
    plt.xlabel('n')
    plt.ylabel('Amplitude')
    plt.legend()
    plt.title('Problem 1e')
    plt.grid(True)
    plt.show()
problem_1e()



# (h) Compute the energy of the signal x[n]

def problem_1h():
    e=0 
    for n in n_values:
        e+=x_n(n)*x_n(n)
    
    print("The energy og the signal x[n] = ",e)
problem_1h()

# Energy = sum of (x[n])^2 for all n


\end{lstlisting}


