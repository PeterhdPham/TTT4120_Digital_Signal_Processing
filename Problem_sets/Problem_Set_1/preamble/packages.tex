%%% The following looks horrible, but essentially sets up biblatex for producing
%%% bibliographies that look nice (you likely will not even need this)
\usepackage[style=authoryear-comp, maxcitenames=2, isbn=false, url=false,
giveninits=true, doi=false, eprint=false, dashed=false, date=year,
related=false, mergedate=true]{biblatex}
\renewbibmacro{in:}{%
  \ifboolexpr{%
     test {\ifentrytype{article}}%
     or
     test {\ifentrytype{inproceedings}}%
  }{}{\printtext{\bibstring{in}\intitlepunct}}%
}
\renewbibmacro*{cite:labelyear+extrayear}{%
  \ifentrytype{online}
    {}
    {\iffieldundef{labelyear}
      {}
      {\printtext[bibhyperref]{%
         \printfield{labelyear}
         \printfield{extrayear}}}}}


\DeclareFieldFormat[article,inbook,incollection,techreport,inproceedings,patent,thesis,unpublished]{title}{#1\isdot}
\DeclareFieldFormat{pages}{#1}

\usepackage{a4wide}
\usepackage[plain]{algorithm} % algorithms package
\usepackage{algpseudocode} % pseudo-code package
\usepackage{amsmath}
\usepackage{amsbsy} % for producing bold maths symbols
\usepackage{amsfonts} % an extended set of fonts for maths
\usepackage{amssymb} % various maths symbols
\usepackage{amsthm} % for producing theorem-like environments
\usepackage{datetime2} % managing dates and times
\usepackage{delimseasy} % makes easy the manual sizing of brackets, square brackets, and curly brackets
\usepackage{enumitem} % customing list environments
\usepackage{extramarks} % extra marks
\usepackage{fancyhdr} % headers and footers
\usepackage{float} % makes dealing with floats (e.g. tables and figures) easier
\usepackage{framed} % for producing framed boxes
\usepackage{graphicx} % for including graphics in the document
\usepackage{hyperref} % automatically produce hyperlinks for cross-references
\usepackage{import}     % Enable importing of sections
\usepackage{mathtools} % package for maths (fixes some deficiences of amsmath so is preferred)
\usepackage{microtype} % better font sizing (extremely helpful with long equations!)
\usepackage{newtx} % a fonts package
\usepackage{pdfpages} % for including pdf documents inside the compiled pdf
\usepackage{pgf} % produce pdf graphics using LaTeX
\usepackage{pgfplots} % create normal/logarithmic plots in two and three dimensions
\pgfplotsset{compat=1.18} % sorts out the compatability warning
\usepackage{physics} % useful for vector calculus and linear algebra symbols
\usepackage[theorems]{tcolorbox} % for producing coloured boxes
\tcbuselibrary{theorems} % theorems with tcolorbox
\usepackage{tikz-3dplot} % for producing 3d plots
\usepackage{tikz} % for drawing graphics in LaTeX
\usepackage{tkz-base} % drawing with a Cartesian coordinate system
\usepackage{tkz-euclide} % drawing in Euclidean geometry
\usepackage{xcolor} % a package for colours
\usepackage{listings}

\usetikzlibrary{automata, positioning}
\DeclareMathAlphabet{\mathcal}{OMS}{cmsy}{m}{n}
