\section{Problem 1 (3 points)}
Two signals $x[n]$ and $y[n]$ are given by


\begin{equation*}
    x[n]=\left\{\begin{array}{ll}2 & n=0 \\ 1 & n= \pm 1 \\ 0 & \text { otherwise, }\end{array} \quad y[n]=\left\{\begin{array}{ll}1 & -M \leq n \leq M \\ 0 & \text { otherwise }\end{array}\right.\right. 
\end{equation*}

\subsection*{(a) Show that the Fourier transform of $x[n]$ is given by... and sketch it for $\omega \in [-\pi , \pi]$.}

\begin{equation*}
    X(\omega)=2+2 \cos{\omega}
\end{equation*}

We have that

\begin{equation*}
    X(\omega)=\sum_{n=-\infty}^\infty x[n]e^{-j\omega n}
\end{equation*}

In our case we can write

\begin{equation*}
    X(\omega)=\sum_{n=-1}^1 x[n]e^{-j\omega n} = -e^{j\omega}+ 2 + e^{-j\omega}
\end{equation*}

From Euler's formula we get that

\begin{equation*}
    \cos{x} = \frac{e^{ix}+e^{-ix}}{2}
\end{equation*}

by combining these two we get:

\begin{equation*}
    X(\omega)=2+2 \cos{\omega}
\end{equation*}

\importimage{1a}

\subsection*{(b) Show that the Fourier transform of y[n] is given by}

\begin{equation*}
    Y(\omega)=\frac{\sin \left(\omega\left(M+\frac{1}{2}\right)\right)}{\sin \left(\frac{\omega}{2}\right)} 
\end{equation*}

\textbf{and sketch it for $M=10$ and $\omega \in [-\pi,\pi]$}

\begin{equation*}
    Y(\omega)=\sum_{n=-M}^M e^{-j\omega n}
\end{equation*}

This is a finite geometric series. By using the general formula for a geometric series
\begin{equation*}
    \sum_{k=m}^{n} a r^{k}=\left\{\begin{array}{ll}a(n-m+1) & \text { if } r=1 \\ \frac{a\left(r^{m}-r^{n+1}\right)}{1-r} & \text { if } r \neq 1\end{array}\right. 
\end{equation*}

we get
\begin{equation*}
    Y(\omega)=\frac{e^{j\omega M}-e^{-j\omega (M+1)}}{1-e^{-j\omega}}
\end{equation*}

by multiplying the numerator and the denominator by $e\frac{j\omega}{2}$.

\begin{equation*}
    \begin{aligned} Y(\omega) & =\frac{e^{j \omega M} e^{j \omega / 2}-e^{-j \omega(M+1)} e^{j \omega / 2}}{\left(1-e^{-j \omega}\right) e^{j \omega / 2}} \\ & =\frac{e^{j \omega\left(M+\frac{1}{2}\right)}-e^{-j \omega\left(M+\frac{1}{2}\right)}}{e^{j \omega / 2}-e^{-j \omega / 2}}\end{aligned}
\end{equation*}

Since we have expresions of the form $e^{j\Theta}-e^{-j\Theta}$, which can be simplidied using Euler's formula

\begin{equation*}
    \begin{array}{c}e^{j \theta}=\cos (\theta)+j \sin (\theta) \\ e^{-j \theta}=\cos (-\theta)+j \sin (-\theta)=\cos (\theta)-j \sin (\theta)\end{array} 
\end{equation*}

Therefore:

\begin{equation*}
    e^{j\Theta}-e^{-j\Theta}= 2j\sin{\Theta}
\end{equation*}

By substituing this back into the equation, we get

\begin{equation*}
    Y(\omega)=\frac{2j\sin \left(\omega\left(M+\frac{1}{2}\right)\right)}{2j\sin \left(\frac{\omega}{2}\right)} 
\end{equation*}


\begin{equation*}
    Y(\omega)=\frac{\sin \left(\omega\left(M+\frac{1}{2}\right)\right)}{\sin \left(\frac{\omega}{2}\right)} 
\end{equation*}

\importimage{1b}

\subsection*{(c) Explain why the signals $x[n]$ and $y[n]$ have real valued spectra.}

The signals have a real valued spectra because the sequencences $x[n]$ and $y[n]$ are real and symmetric.

\subsection*{(d) Let the signal}

\begin{equation*}
    z[n]=\sum_{l=-\infty}^{\infty} x[n-l N] 
\end{equation*}

\textbf{be the periodic extension of $x[n]$. Assume that N is greater than the length of the signal, i.e. $N > 3$.\\Sketch the signal $z[n]$.\\ Find the Fourier series coefficients $\{c_k\}$ of $z[n]$.\\ Sketch $\{c_k\}$ as a function of $\omega=2\pi \frac{k}{N}\in [-\pi,\pi]$ for $N=10$.}

For a discrete-time periodic signal, the Fourier Series coefficients $\{c_k\}$ can be defined as:

\begin{equation*}
    c_{k}=\frac{1}{N} \sum_{n=0}^{N-1} z[n] e^{-j(2 \pi k n) / N} 
\end{equation*}

For $z[n]$, you can see that most terms in this sum will be zero, except for the few terms that correspond to the non-zero values of $x[n]$. Thus:

\begin{equation*}
    c_{k}=\frac{1}{N}\left(2+e^{-j(2 \pi k) / N}+e^{j(2 \pi k) / N}\right)
\end{equation*}

Simplifying this using Euler's formula $ e^{j \theta}+e^{-j \theta}=2 \cos (\theta) $, we get:

\begin{equation*}
    \begin{array}{c}c_{k}=\frac{1}{N}\left(2+2 \cos \left(\frac{2 \pi k}{N}\right)\right) \\c_{k}=\frac{2}{N}\left(1+\cos \left(\frac{2 \pi k}{N}\right)\right)\end{array}
\end{equation*}

\importimage{1d}

\subsection*{(e) Compare the spectra of $x[n]$ and $z(n)$, i.e. $X(\omega)$ and $\{c_k\}$. What is the
relationship between the spectra?}

$ X(\omega) $ and $ \left\{c_{k}\right\} $ are essentially two different ways of representing the frequency content of a signal. $ X(\omega) $ is used for the original aperiodic $ x[n] $, while $ \left\{c_{k}\right\} $ is used for the periodic $ z[n] $, which is formed by periodically repeating $ x[n] $.