\section{Problem 3 (3 points)}
\textbf{Two systems (from Problem Set 1) are given by the following difference
equations}

\begin{equation*}
    \begin{array}{l}y[n]=x[n]+2 x[n-1]+x[n-2] \\ y[n]=-0.9 y[n-1]+x[n] .\end{array} 
\end{equation*}

\subsection*{(a) Find the frequency responses of these two systems.}

To find the frequency response $ H(\omega) $ of a system described by a difference equation, you can take the Fourier Transform of both sides of the equation and solve for $ H(\omega) $, which is essentially $ \frac{Y(\omega)}{X(\omega)} $.

\subsubsection*{For the first system:}

The difference equation is:
\begin{equation*}
y[n] = x[n] + 2x[n-1] + x[n-2]
\end{equation*}
Taking the Fourier Transform of both sides:
\begin{equation*}
Y(\omega) = X(\omega) + 2e^{-j\omega}X(\omega) + e^{-2j\omega}X(\omega)
\end{equation*}
Simplifying:
\begin{equation*}
Y(\omega) = X(\omega)(1 + 2e^{-j\omega} + e^{-2j\omega})
\end{equation*}
The frequency response $ H_1(\omega) $ is:
\begin{equation*}
H_1(\omega) = \frac{Y(\omega)}{X(\omega)} = 1 + 2e^{-j\omega} + e^{-2j\omega}
\end{equation*}

\subsubsection*{For the second system:}

The difference equation is:
\begin{equation*}
y[n] = -0.9y[n-1] + x[n]
\end{equation*}
Taking the Fourier Transform of both sides:
\begin{equation*}
Y(\omega) = -0.9e^{-j\omega}Y(\omega) + X(\omega)
\end{equation*}
Solving for $ Y(\omega) $:
\begin{equation*}
Y(\omega) = \frac{X(\omega)}{1 + 0.9e^{-j\omega}}
\end{equation*}
The frequency response $ H_2(\omega) $ is:
\begin{equation*}
H_2(\omega) = \frac{Y(\omega)}{X(\omega)} = \frac{1}{1 + 0.9e^{-j\omega}}
\end{equation*}

\subsection*{Find the magnitude and phase responses of the two systems. Are they
even or odd functions?}

To find the magnitude and phase of a frequency response $H(\omega)$, we can express $H(\omega)$ in polar form: $H(\omega)=|H(\omega)|e^{j\angle H(\omega)}$ where $|H(\omega)|$ is the magnitude and $\angle H(\omega)$ is the phase.

\subsubsection*{System 1}

The difference equation is $ y[n] = x[n] + 2x[n-1] + x[n-2] $.

The Fourier Transform of the system, known as the frequency response $ H_1(\omega) $, is given by:
\begin{equation*}
H_1(\omega) = 1 + 2e^{-j\omega} + e^{-2j\omega}
\end{equation*}
To find the magnitude $ |H_1(\omega)| $:
\begin{equation*}
\begin{aligned}
|H_1(\omega)| &= \sqrt{\text{Re}[H_1(\omega)]^2 + \text{Im}[H_1(\omega)]^2} \\
&= \sqrt{(1 + 2\cos(\omega) + \cos(2\omega))^2} \\
&= \sqrt{2 + 4\cos(\omega) + 4\cos^2(\omega) + 2\cos(2\omega)} \\
&= \sqrt{2(1 + 2\cos(\omega) + 2\cos^2(\omega) + \cos(2\omega))}
\end{aligned}
\end{equation*}
To find the phase $ \angle H_1(\omega) $:
\begin{equation*}
\angle H_1(\omega) = 0
\end{equation*}

To determine whenever the function is even or odd

\begin{equation*}
    H_{1}(-\omega)=1+2 e^{j \omega}+e^{2 j \omega}
\end{equation*}

$ H_{1}(\omega) $ and $ H_{1}(-\omega) $ are not equal, nor are $ H_{1}(\omega) $ and $ -H_{1}(-\omega) $- Thus, $ H_{1}(\omega) $ is neither even nor odd.2.

\subsubsection*{System 2}

The difference equation is $ y[n] = -0.9y[n-1] + x[n] $.

The frequency response $ H_2(\omega) $ is:
\begin{equation*}
H_2(\omega) = \frac{1}{1 + 0.9e^{-j\omega}}
\end{equation*}
To find the magnitude $ |H_2(\omega)| $:
\begin{equation*}
\begin{aligned}
|H_2(\omega)| &= \left| \frac{1}{1 + 0.9(\cos(-\omega) - j\sin(-\omega))} \right| \\
&= \frac{1}{\sqrt{(1 - 0.9\cos(\omega))^2 + (0.9\sin(\omega))^2}} \\
&= \frac{1}{\sqrt{1 - 1.8\cos(\omega) + 0.81}}
\end{aligned}
\end{equation*}
To find the phase $ \angle H_2(\omega) $:
\begin{equation*}
\angle H_2(\omega) = -\text{atan2}(0.9 \sin(\omega), 1 - 0.9 \cos(\omega))
\end{equation*}

To determine whenever the function is even or odd

\begin{equation*}
    H_{2}(-\omega)=\frac{1}{1+0.9 e^{j \omega}} 
\end{equation*}
 $ H_{2}(\omega) $ and $ H_{2}(-\omega) $ are not equal, nor are $ H_{2}(\omega) $ and $ -H_{2}(-\omega) $- Thus, $ H_{2}(\omega) $ is neither even nor odd.

 \subsection*{(c) Use Python (the functions scipy.signal.freqz, numpy.abs and numpy.angle) to find
 and plot the magnitude and phase responses of the systems.}

 \importimage{3c1}
 \importimage{3c2}
 
 \subsection*{(d) Determine whether each system represents a lowpass, bandpass,
 bandstop or highpass filter. Justify your answers.}
 From the plots, we can se that system 1 is a lowpass filter, while system 2 is a highpass filter. This is because the amplitude is high at a low frequency at system 1 and opposite at system 2.

 \subsection*{The signal $ x[n]=\frac{1}{2} \cos \left(\frac{\pi}{2} n+\frac{\pi}{4}\right) $ is passed through the two systems. Find the frequency, amplitude and phase of the corresponding output signals.}
The signal $ x[n]=\frac{1}{2} \cos \left(\frac{\pi}{2} n+\frac{\pi}{4}\right) $ can be represented as

\begin{equation*}
     x[n]=\frac{1}{2}\left(e^{j\left(\frac{\pi}{4}\right)} e^{j\left(\frac{\pi}{2} n\right)}+e^{-j\left(\frac{\pi}{4}\right)} e^{-j\left(\frac{\pi}{2} n\right)}\right) 
\end{equation*}

 The Fourier Transform $ X(\omega) $ of $ x[n] $ is:

 \begin{equation*}
 X(\omega) = \frac{1}{4} \left( e^{j\left(\frac{\pi}{4}\right)} \delta\left(\omega-\frac{\pi}{2}\right) + e^{-j\left(\frac{\pi}{4}\right)} \delta\left(\omega+\frac{\pi}{2}\right) \right)
 \end{equation*}
 \subsubsection*{For the first system $ y_1[n] = x[n] + 2x[n-1] + x[n-2] $}

The frequency response $ H_1(\omega) $ is:

\begin{equation*}
H_1(\omega) = 1 + 2e^{-j\omega} + e^{-j2\omega} = 1 + 2 \cos(\omega) - j2 \sin(\omega) + \cos(2\omega) - j \sin(2\omega)
\end{equation*}



Thus, the output $ Y_1(\omega) $ will be:

\begin{equation*}
Y_1(\omega) = X(\omega) \cdot H_1(\omega)
\end{equation*}

\begin{equation*}
    Y_{1}(\omega)=\frac{1}{4}\left(e^{j\left(\frac{\pi}{4}\right)} H_{1}\left(\frac{\pi}{2}\right) \delta\left(\omega-\frac{\pi}{2}\right)+e^{-j\left(\frac{\pi}{4}\right)} H_{1}\left(-\frac{\pi}{2}\right) \delta\left(\omega+\frac{\pi}{2}\right)\right) 
\end{equation*}


\subsubsection*{For the second system $ y_2[n] = -0.9y_2[n-1] + x[n] $}

The frequency response $ H_2(\omega) $ is:

\begin{equation*}
H_2(\omega) = \frac{1}{1 + 0.9 e^{-j\omega}} = \frac{1}{1 + 0.9 \cos(\omega) - j0.9 \sin(\omega)}
\end{equation*}

The output $ Y_2(\omega) $ will then be:

\begin{equation*}
Y_2(\omega) = X(\omega) \cdot H_2(\omega)
\end{equation*}

\begin{equation*}
    Y_{2}(\omega)=\frac{1}{4}\left(e^{j\left(\frac{\pi}{4}\right)} H_{2}\left(\frac{\pi}{2}\right) \delta\left(\omega-\frac{\pi}{2}\right)+e^{-j\left(\frac{\pi}{4}\right)} H_{2}\left(-\frac{\pi}{2}\right) \delta\left(\omega+\frac{\pi}{2}\right)\right) 
\end{equation*}
