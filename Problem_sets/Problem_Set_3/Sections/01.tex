\section{Problem 1 (2.5 points)}
Consider the two causal analog filters shown in Figure \ref{fig:analog_filters.png}.
\importimagewcaption{analog_filters.png}{Analog filters}

\subsection*{(a) Derive the differential equations and transfer functions for the two filters.}

For the RC filter

\begin{align*}
    v_R(t)&=Ri(t)\\
    i(t)&=C\frac{dv(t)}{dt}\\
    v_R(t)&=RC\frac{dv(t)}{dt}\\
\end{align*}

using KVL we get

\begin{align*}
    x(t)&=v_R(t)+v_C(t)\\
    x(t)&=RC\frac{dv(t)}{dt}+v_C(t)\\
    y(t)&=x(t)-RC\frac{dv(t)}{dt}
\end{align*}

The transfer function $H(s)=\frac{Y(s)}{X(s)}$ 

\begin{align*}
    \mathcal{L}\left\{ x(t)\right\} &= R C \cdot \mathcal{L}\left\{ \frac{dy(t)}{dt}\right\}+\mathcal{L}\left\{ y(t)\right\} \\
    X(s) &= R C s Y(s) + Y(s)
\end{align*}

This gives us

\begin{align*}
    H(s)&=\frac{Y(s)}{X(s)}=\frac{Y(s)}{R C s Y(s) + Y(s)}\\
    H(s)&=\frac{1}{RCs+1}
\end{align*}

For the RL filter

\begin{align*}
    v_R(t)&=Ri(t)\\
    v_L(t)&=L\frac{di(t)}{dt}
\end{align*}

using KVL we get

\begin{align*}
    x(t)&=Ri(t)+L\frac{di(t)}{dt}\\
    y(t)&=x(t)-Ri(t)
\end{align*}


To find the transfer function we need to take the laplace transform:

\begin{align*}
    \mathcal{L}\left\{ x(t)\right\} &= R\mathcal{L}\left\{i(t)\right\}+L\mathcal{L}\left\{ \frac{di(t)}{dt})\right\} \\
    X(s) &= RI(s)+ LsI(s)\\
    \mathcal{L}\left\{ y(t)\right\} &= \mathcal{L}\left\{x(t)\right\}-R\mathcal{L}\left\{i(t)\right\} \\
    Y(s) &= LsI(s)+RI(s)-RI(s) 
\end{align*}

\begin{align*}
    H(s)&=\frac{Y(s)}{X(s)}=\frac{LsI(s)}{RI(s)+ LsI(s)}\\
    H(s)&=\frac{LsI(s)}{L  s I(s) + RI(s)}\\
    H(s)&=\frac{Ls}{L s + R}
\end{align*}

\subsection*{(b) Determine frequency responses and filter types (i.e. lowpass, highpass, bandpass, bandstop) for the filters.}

For the RL filter the capacitor will ground all high frequencies, therefore it is a lowpass filter. The RL filter will ground all low frequencies, therefore it is a highpass filter.

\subsection*{(c) Derive the unit pulse responses for the filters.}
We can derive the unit pulse response for the filter by taking the inverse Laplace transform of the transfer function, $H(s)$. We can use this by using a inverce laplace table.

For the RC filter we have

\begin{equation*}
    \frac{1}{s-a}=e^{at}
\end{equation*}

\begin{equation*}
    h(t)=\mathcal{L}^{-1}\left\{ H(s)\right\} =e^{-\frac{t}{RC}}u(t)
\end{equation*}

For the RL filter we have:

\begin{align*}
    H(s)&=\frac{Ls}{L s + R}\\
    &=1-\frac{R}{L s + R}\\
    &=1-\frac{1}{\frac{R}{L} s+1}
\end{align*}

this using the same equation from the table we get

\begin{equation*}
    h(t)=\delta(t)- e^{-\frac{tR}{L}}u(t)
\end{equation*}