\section{Problem 4 (2 points)}
\textbf{A digital filter is given by the following difference equation}
\begin{equation*}
    y[n]=x[n]-x[n-2]-\frac{1}{4} y[n-2]
\end{equation*}

\subsection*{(a) Find the transfer function of the filter.}
The transfer function \( H(z) \) can be found by taking the Z-transform of the given difference equation:
\begin{equation*}
    Y(z) = X(z) - z^{-2}X(z) - \frac{1}{4}z^{-2}Y(z)
\end{equation*}
Rearranging terms to solve for \( \frac{Y(z)}{X(z)} \):
\begin{equation*}
    H(z) = \frac{Y(z)}{X(z)} = \frac{1 - z^{-2}}{1 + \frac{1}{4}z^{-2}}
\end{equation*}
Simplifying:
\begin{equation*}
    H(z) = \frac{z^2 - 1}{z^2 + 0.25}
\end{equation*}

\subsection*{(b) Find the poles and zeros of the filter and sketch them in the z-plane.}
The zeros of the transfer function are at \( z = -1 \) and \( z = 1 \).
The poles of the transfer function are at \( z = 0.5i \) and \( z = -0.5i \).
The poles and zeros are sketched in the z-plane as shown in the figure attached.

\importimage{4b.png}


\subsection*{(c) Is the filter stable? Justify your answer based on the pole-zero plot.}
The filter is stable as all the poles are inside the unit circle in the z-plane. 

\subsection*{(d) Determine the filter type (i.e. HP, LP, BP or BS) based on the pole-zero plot.}
The filter is a Bandpass (BP) filter. This can be inferred from the pole-zero plot as there are zeros at both \( z = 1 \) and \( z = -1 \), suggesting that it attenuates both the high and low-frequency components, allowing a certain band of frequencies to pass through.

