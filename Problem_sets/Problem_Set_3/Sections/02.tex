\section{Problem 2 (2.5 points)}
\textbf{Find the region of convergence and unit pulse response $h[n]$ for the following digital filters.}
\subsection*{(a) A causal filter with transfer function}
\begin{equation*}
     H(z)=\frac{1}{1-\frac{2}{3} z^{-1}} 
\end{equation*}

We start by finding the pole

\begin{align*}
    1-\frac{2}{3} z^{-1}&=0\\
    z&=\frac{2}{3}
\end{align*}

Since this is a causal sequence, the ROC will be the region outside the circle with a radius equal to the magnitude of the largest pole which in this case is $|z|>\frac{2}{3}$

To find the unit pulse response we can use a simple inverse z-transform table

\begin{equation*}
    \frac{1}{1-az^{-1}}=a^nu[n]
\end{equation*}

this gives us

\begin{equation*}
    h[n]=\left(\frac{2}{3}\right)^nu[n]
\end{equation*}

\subsection*{(b) A causal filters with transfer function}

\begin{equation*}
    H(z)=\frac{1}{\left(1+\frac{1}{2} z^{-1}\right)\left(1-z^{-1}\right)} 
\end{equation*}

To find the poles of the transfer function, we set the denominator to zero and solve for \( z \):

\begin{align*}
    \left(1+\frac{1}{2} z^{-1}\right)\left(1-z^{-1}\right) &= 0 \\
    \Rightarrow \quad 1+\frac{1}{2} z^{-1} &= 0 \quad \text{or} \quad 1 - z^{-1} = 0
\end{align*}

Solving the first equation \(1 + \frac{1}{2} z^{-1} = 0\):

\begin{align*}
    \frac{1}{2} z^{-1} &= -1 \\
    z^{-1} &= -2 \\
    z &= -\frac{1}{2} \quad \text{(Pole 1)}
\end{align*}

Solving the second equation \(1 - z^{-1} = 0\):

\begin{align*}
    z^{-1} &= 1 \\
    z &= 1 \quad \text{(Pole 2)}
\end{align*}

Thus, the transfer function has poles at \( z = -\frac{1}{2} \) and \( z = 1 \).

Since the magnitde of the largest pole is 1 the ROC will be the region outside the circle with a radius equal to $1$.

To find the impulse response, we need to perform partial fraction decomposition.

\begin{align*}
    H(z)&=\frac{1}{\left(1+\frac{1}{2} z^{-1}\right)\left(1-z^{-1}\right)}=\frac{z}{\left(z+\frac{1}{2}\right)\left(z-1\right)}\\
    &=\frac{A}{\left(z+\frac{1}{2}\right)}+\frac{B}{\left(z-1\right)}
\end{align*}

\begin{align*}
    z&=A(z-1)+B(z+\frac{1}{2})\\
    0&=z-\left(A(z-1)+B\left(z+\frac{1}{2}\right)\right)\\
\end{align*}

this gives us $A=\frac{1}{3}$ and $B=\frac{2}{3}$

This gives us the inverse Z-transform

\begin{equation*}
    \frac{A}{1-az^{-1}}=A a^nu[n]
\end{equation*}

\begin{equation*}
    h[n]= \frac{1}{3} \left(-\frac{1}{2}\right)^n u[n]+\frac{2}{3}u[n]
\end{equation*}

\subsection*{(c) An anti-causal filter with transfer function}
\begin{equation*}
    H(z)=\frac{z^{-1}}{\left(1+\frac{3}{2} z^{-1}\right)\left(1-3 z^{-1}\right)} 
\end{equation*}

The poles of this filter is at $z=-\frac{3}{2}$ and $z=3$. Since this is an anti-causal filter, the ROC will be inside a circle with the radius equal to the magnitude of the smallest pole $\frac{3}{2}$

To find the impulse response, we need to perform partial fraction decomposition.
\begin{align*}
    H(z)&=\frac{z^{-1}}{\left(1+\frac{3}{2} z^{-1}\right)\left(1-3 z^{-1}\right)}=\frac{1}{\left(z+\frac{3}{2}\right)\left(z-3\right)}\\
    H(z)&=\frac{A}{\left(z+\frac{3}{2}\right)}+\frac{B}{\left(z-3\right)}
\end{align*}

\begin{align*}
    1&=A\left(z-3\right)+B\left(z+\frac{3}{2}\right)\\
\end{align*}

This gives us $A=-\frac{2}{9}$ and $B=\frac{2}{9}$

This gives us the inverse Z-transform

\begin{equation*}
    \frac{1}{1-az^{-1}}=-a^nu[-n-1]
\end{equation*}

\begin{equation*}
    h[n] = -\frac{2}{9} \left(-\frac{3}{2}\right)^n u[-n-1] + \frac{2}{9} 3^n u[-n-1]
\end{equation*}

\subsection*{(d) Which of the three systems are stable? Justify your answer.}

The stability of a LTI system can be determined by the locations of the poles of its transfer function. The magnitudes of all the poles should be less than 1. Therefore the first system is the only stable one as the magnitude of the pole is less than 1.