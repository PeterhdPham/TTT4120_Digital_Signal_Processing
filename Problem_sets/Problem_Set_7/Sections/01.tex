\section{Problem 1 (4 points)}

\textbf{Given three different types of white noise with unit variance:}
\begin{itemize}
    \item White binary noise (each sample takes the value -1 or 1 with the same probability)
    \item White Gaussian noise (each sample has a normal distribution)
    \item White uniform noise
\end{itemize}

\subsection*{(a) For each noise type, generate and plot one realization of length 100 samples. (Useful Matlab functions: rand and randn)}
\begin{itemize}
    \item What do these noises have in common?
    \item How do they differ?
\end{itemize}

We start by describing the noises individualy:

\subsubsection*{White binary noise:}
Binary noise consist of a signal consisting of only binary values, this means that the signal is either 1 or 0, when you have a long sequence of random sequence of 1s and 0s you will end up with a signal where all frequencies are equally represented.
\importimagewcaptionw{1aBinary.png}{Plot of white binary noise with a length of 100}{1}
\importimagewcaptionw{1aBinaryDistro.png}{Plot of the distrobution of the white binary noise with a length of 100}{1}

\subsubsection*{White Gaussian noise:}
This noise has amplitude values that are statistically distributed according to the Gaussian distrobution. It is a white noise because it has a flat spectrum, meaning all frequencies are equally likely
\importimagewcaptionw{1aGaussian.png}{Plot of white Gaussian noise with a length of 100}{1}
\importimagewcaptionw{1aGaussianDistro.png}{Plot of the distrobution of the white Gaussian noise with a length of 100}{1}

\subsubsection*{White uniform noise:}
This noise has a amplitude value that are distributed uniformily across a vertain range
\importimagewcaptionw{1aUniform.png}{Plot of white uniform noise with a length of 100}{1}
\importimagewcaptionw{1aUniformDistro.png}{Plot of the distrobution of the white uniform noise with a length of 100}{1}
We can se in the figures \ref{fig:1aBinary.png}, \ref{fig:1aGaussian.png} and \ref{fig:1aUniform.png} That what they have in common is the sqeuence of positive and negative values are random, this is what gives the signals its "white" noise features.

They differ in terms of amplitude, from figure \ref{fig:1aBinaryDistro.png} we can si that binary are either 1 or 0, \ref{fig:1aGaussianDistro.png} we can se that the the distrobution of amplitude is normalised and in figure \ref{fig:1aUniformDistro.png} we can se that the distrobution of amplitude is wait for it.... uniform! After playing a longer sequence of the diffrent noises, I was also able to observe that the binary was the loudest, then Gaussian, and then uniform being the most quiet. This was with binary having a random choice between 1 and -1, Gaussian having a normal distrobution with mean being 0 and standard deviation being 1, and lastly uniform having a uniform distrobution between 1 and -1.

\subsection*{(b) For each noise type}
\begin{itemize}
    \item write an expression for the probability distribution
    \item compute the mean value, autocorrelation function and power density spectrum
\end{itemize}

\subsubsection*{Probability distrobution of white binary noise:}
$$
P_x=\left(\begin{array}{l}
n \\ x
\end{array}\right) p^x q^{n-x}
$$
\subsubsection*{Probability distrobution of white Gaussian noise:}
$$f(x)=\frac{1}{\sigma \sqrt{2\pi}}e^{-\frac{1}{2}\frac{x-\mu}{\sigma}}$$
where $\sigma$ is the standard deviation, $\mu$ is mean and $x$ being the amplitude.
\subsubsection*{Probability distrobution of white uniform noise:}
$$f(x)=\frac{1}{b-a}$$
where $a$ is the minimum amplitude and $b$ is the maximum amplitude

When computing the mean value of the diffrent noises, I ended up with values approximately to 0:
\begin{itemize}
    \item Mean Value of Binary White Noise: -0.08
    \item Mean Value of White Gaussian Noise: 0.11586957047161961
    \item Mean Value of White Uniform Noise: 0.03997036015289663
\end{itemize}
This is expected as a ideal mean for white noise should be zero

When plotting the autocorrelation we got:
\importimagewcaptionw{1bAutocorrelationBinary.png}{Autocorrelation of binary white noise}{1}
\importimagewcaptionw{1bAutocorrelationGaussian.png}{Autocorrelation of Gaussian white noise}{1}
\importimagewcaptionw{1bAutocorrelationUniform.png}{Autocorrelation of uniform white noise}{1}


When plotting the power deinsity spectrum we got:
\importimagewcaptionw{1bPowerDensitySpectrumBinary.png}{Power density spectrum of }{1}
\importimagewcaptionw{1bPowerDensitySpectrumGaussian.png}{Power density spectrum of Gaussian white noise}{1}
\importimagewcaptionw{1bPowerDensitySpectrumUniform.png}{Power density spectrum of uniform white noise}{1}

\subsection*{(c) We would now like to estimate the statistical properties of the three noise types based on a noise segment.}
\begin{itemize}
    \item Generate a segment of 20000 samples for each noise type.
    \item Compute mean value estimates and compare to the theoretical values computed in (b).
    \item Compute estimates of the autocorrelation function. Plot them on the interval $[-10,10]$, and compare to the theoretical values computed in (b).
\end{itemize}



\importimagewcaptionw{1cBinary.png}{Plot of white binary noise with a length of 20000}{1}
\importimagewcaptionw{1cBinaryDistro.png}{Plot of the distrobution of the white binary noise with a length of 20000}{1}
\importimagewcaptionw{1cGaussian.png}{Plot of white Gaussian noise with a length of 20000}{1}
\importimagewcaptionw{1cGaussianDistro.png}{Plot of the distrobution of the white Gaussian noise with a length of 20000}{1}
\importimagewcaptionw{1cUniform.png}{Plot of white uniform noise with a length of 20000}{1}
\importimagewcaptionw{1cUniformDistro.png}{Plot of the distrobution of the white uniform noise with a length of 20000}{1}

When computing the mean value of the diffrent noises we got:
\begin{itemize}
    \item Mean Value of Binary White Noise: -0.0036
    \item Mean Value of White Gaussian Noise: 0.0025846854929424063
    \item Mean Value of White Uniform Noise: 0.00016222991807518508
\end{itemize}
This is also expected as a longer sequence will be able have "more random" sequences and therefor be better at represent all the frequencies 

When plotting the autocorrelation we got:
\importimagewcaptionw{1cAutocorrelationBinary.png}{Autocorrelation of binary white noise}{1}
\importimagewcaptionw{1cAutocorrelationGaussian.png}{Autocorrelation of Gaussian white noise}{1}
\importimagewcaptionw{1cAutocorrelationUniform.png}{Autocorrelation of uniform white noise}{1}

We se that the autocorrelation is greater for a higher sequence, this is because there is a greater chance of a similar sequence more places on on the signal when the sequence length increases.

When plotting the power deinsity spectrum we got:
\importimagewcaptionw{1cPowerDensitySpectrumBinary.png}{Power density spectrum of }{1}
\importimagewcaptionw{1cPowerDensitySpectrumGaussian.png}{Power density spectrum of Gaussian white noise}{1}
\importimagewcaptionw{1cPowerDensitySpectrumUniform.png}{Power density spectrum of uniform white noise}{1}