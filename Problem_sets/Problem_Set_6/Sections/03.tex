\section{Problem 3: Spectral Estimation [2.5 points]}
\textbf{In this problem we will use DFT to estimate the spectrum of a discrete-time signal based on finite signal segments. The following signal is considered}
$$
x(n)=\sin \left(2 \pi f_1 n\right)+\sin \left(2 \pi f_2 n\right),
$$
\textbf{where $f_1=7 / 40$ and $f_2=9 / 40$.}
\subsection*{(a) Sketch the magnitude spectrum of the sampled signal, $|X(f)|$ for $f \in[0,0.5]$.}

\importimagewcaption{3a.png}{magnitude spectrum of the sampled signal, $|X(f)|$ for $f \in[0,0.5]$.}

\subsection*{(b) Use Matlab to generate a segment of length 100 of the signal $x(n)$.}
\importimagewcaption{3b.png}{Segment of length 100 of the signal $x(n)$.}


\subsubsection*{Use DFT of length 1024 to estimate the spectrum $X(f)$ based on this signal segment.}
\importimagewcaption{3b2.png}{DFT of length 1024 to estimate the spectrum $X(f)$.}


\subsubsection*{Plot the estimated magnitude spectrum for $f \in[0,0.5]$.}


\subsubsection*{Repeat the above with segment lengths equal to 1000,30 and 10 .}
\importimagewcaption{3b3.png}{Plot of the estimated magnitude spectrum for $f \in[0,0.5]$ with segment lengths equal to 1000,30 and 10.}


\subsubsection*{Compare with the sketch in (a) and explain the similarities and differences.}
It does look wuite similar as in (a) wt an increased N, We can se that at a very high segment lenght we can se that $f_1$ is more dominant than $f_2$

\subsection*{(c) Repeat (b) with segment of length 100 of the signal $x(n)$, using DFT lengths equal to 256 and 128 .
What effect does the DFT length have on spectral estimation.}

\importimagewcaption{3c.png}{Plot of the estimated magnitude spectrum for $f \in[0,0.5]$using DFT lengths equal to 256 and 128}

As the DFT length increases, the frequency resolution improves. In the plots, you can observe that the spectrum with DFT length 256 provides a more detailed representation compared to the one with DFT length 128.